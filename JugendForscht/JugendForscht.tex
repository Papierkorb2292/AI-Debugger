\documentclass[a4paper,12pt,ngerman]{scrartcl}
\usepackage{babel}
\usepackage{mathptmx}
\usepackage{titlesec}
\usepackage[T1]{fontenc}
\usepackage[utf8x]{inputenc}
\usepackage[a4paper,lmargin=2.5cm,rmargin=2.5cm,tmargin=2.0cm,bmargin=2.0cm,footskip=0.5cm]{geometry}
\usepackage{amsmath}
\usepackage{amssymb}
\usepackage{graphicx}
\usepackage{hyperref}

\setkomafont{sectionentry}{\normalfont}

\titleformat{\section}
  {\normalfont\Large\bfseries}
  {\thesection}{1em}{}

\graphicspath{ {.} }

\renewcommand{\baselinestretch}{1.5} 

\begin{document}
\begin{titlepage}
	Projekttitel: Debuggen mit KI
	\vspace{1cm}
	Teilnehmende: PLACEHOLDER
	
	Erarbeitungsort: ?
	
	Projektbetreuende: PLACEHOLDER
	
	Thema des Projekts: Kann eine KI beim Debuggen eines Programmes helfen?
	
	Fachgebiet: Informatik
	
	Wettbewerbsparte: Jugend forscht
	
	Bundesland: PLACEHOLDER
	
	Wettbewerbsjahr: 2024
	
	\vspace{2cm}
	\vfill
\end{titlepage}
\clearpage
{\normalfont\tableofcontents}
\clearpage

\section{Fachliche Kurzfassung}

\section{Motivation und Fragestellung}

\section{Hintergrund und theoretische Grundlagen}

\section{Vorgehensweise, Materialien und Methoden}

\section{Ergebnisse}

\section{Ergebnissdiskussion}

\section{Fazit und Ausblick}

\section{Quellen- und Literaturverzeichnis}

TODO: Quellen zur Extension Entwicklung (Guide und VSCode Debug Client Source)

BEISPIEL: \url{https://de.wikipedia.org/}, besucht am 01.01.1970

\section{Unterstützungsleistungen}

\end{document}